\begin{abstractpage}
\begin{abstract}{polish}


W ramach niniejszej pracy zrealizowano system detekcji obiektów -- potencjalnych przeszkód dla robota mobilnego (drona) z~wykorzystaniem kamery zdarzeniowej (ang. \textit{Dynamic Vision Sensor (DVS)}). Czujnik ten wykorzystano w~celu zapewnienia poprawnego działania w różnych warunkach oświetleniowych. Przeanalizowano sposoby rozwiązania zagadnienia w~innych projektach dostępnych w literaturze i~na tej podstawie stworzono własny algorytm detekcji. Przygotowano symulację SiL (ang. \textit{Software in the Loop}) umożliwiającą testowanie algorytmu w~świetle dziennym oraz~w nocnych warunkach, oraz tworzenie różnorodnych scen z~przeszkodami. Umieszczono w~niej drona wyposażonego w~symulowaną kamerę zdarzeniową. Przeprowadzono testy systemu wykrywania przeszkód, wykorzystując metodę SiL oraz gotowe zbiory danych z~DVS. Przedstawiono i~omówiono ich wyniki.
% System przetestowano w symulacji SiL (ang. Software-in-the-Loop) oraz na danych rzeczywistych kamer zdarzeniowych. 

%, a następnie przeniesiono na wbudowaną platformę obliczeniową - eGPU Jetson. Dodatkowo, podjęto próbę implementacji systemu na platformie FPGA i uruchomienia na rzeczywistym pojeździe.



\end{abstract}

% \newpage

\begin{abstract}{english}

In this work an object detection system -- obstacles for a~mobile robot (drone) using an event camera (DVS -- Dynamic Vision Sensor) was created. This sensor was applied to ensure correct operation under different lighting conditions. Ways of solving the issue in other designs available in the literature were analysed and a~custom detection algorithm was created. A~SiL (Software in the Loop) simulation was prepared to test the algorithm in daylight and night conditions in variety of scenes with obstacles. A~drone equipped with a simulated event camera was deployed in the scene of testing environment. Tests of the obstacle detection system were carried out using the SiL method and available datasets with DVS data. Their results were presented and discussed.

\end{abstract}
\end{abstractpage}