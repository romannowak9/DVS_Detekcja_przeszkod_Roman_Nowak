\documentclass[12pt]{aghdpl}
% \documentclass[language=en,11pt]{aghdpl}  % praca w języku angielskim

%---------------------------------------------------------------------------

\author{Roman Nowak}
\shortauthor{Roman Nowak}

%\titlePL{Przygotowanie bardzo długiej i pasjonującej pracy dyplomowej w~systemie~\LaTeX}
%\titleEN{Preparation of a very long and fascinating bachelor or master thesis in \LaTeX}

\titlePL{System do detekcji przeszkód zrealizowany z wykorzystaniem kamery zdarzeniowej}
\titleEN{Obstacles detection system using dynamic vision sensor}


\shorttitlePL{Detekcja przeszkód} % skrócona wersja tytułu jeśli jest bardzo długi
\shorttitleEN{DVS obstacles detection}


% Dopuszczalne wartości[1,2]:
% * "Projekt dyplomowy" - na koniec studiów I stopnia
% * "Praca dyplomowa" - na koniec studiów II stopnia
% [1] Zasady dyplomowania w roku akademickim 2020/2021 (Decyzja Dziekana WEAIiIB nr 16/2020 z dnia 9 grudnia 2020 roku)
% [2] Załącznik nr 1a) do Decyzji nr 16/2020 Dziekana Wydziału EAIiIB z dnia 09 grudnia 2020 r.
\thesistype{Projekt dyplomowy}
%\thesistype{Master of Science Thesis}

\supervisor{dr Tomasz Kryjak}
%\supervisor{Paweł Kłeczek, PhD}

\degreeprogramme{Automatyka i Robotyka}
%\degreeprogramme{Automatics and Robotics}

\date{2024}

%\department{Katedra Informatyki Stosowanej}
%\department{Department of Applied Computer Science}

\faculty{Wydział Elektrotechniki, Automatyki, Informatyki i Inżynierii Biomedycznej}
%\faculty{Faculty of Electrical Engineering, Automatics, Computer Science and Biomedical Engineering}

\acknowledgements{Serdecznie dziękuję \dots tu ciąg dalszych podziękowań np. dla promotora, żony, sąsiada itp.}


\begin{document}

	\titlepages
	\RedefinePlainStyle
	
	\setcounter{tocdepth}{2}
	\tableofcontents
	\clearpage

	\chapter{Wprowadzenie}
\label{cha:wprowadzenie}

%---------------------------------------------------------------------------

\section{Cele pracy}
\label{sec:celePracy}


Tu krótko o celach projektu, kolejnych etapach, zamierzonym efekcie końcowym.


%---------------------------------------------------------------------------

\section{Zawartość pracy}

Informacje o zawartości poszczególnych rozdziałów, wykorzystanej bibliografii itd.
\label{sec:zawartoscPracy}

	\chapter{Wstęp teoretyczny}
\label{cha:wstep}

Informacje teoretyczne o wykorzystanych narzędziach, technologiach itp..

%---------------------------------------------------------------------------

\section{Kamery zdarzeniowe}
\label{sec:kamery_zdarzeniowe}

\subsection{Budowa i działanie}

\subsection{Zastosowanie}

\subsection{Rola w projekcie}

\section{Isaac SIM}
\label{sec:isaac_sim}

\subsection{Wprowadzenie}

\subsection{Wykorzystanie w projekcie}

\section{Hardware in the Loop}
\label{sec:hil}

\subsection{Opis metody HIL}

\subsection{Dlaczego ten sposób testowania?}

\section{Układy FPGA}
\label{sec:fpga}

\subsection{Budowa i działanie - podstawy}

\subsection{Przewaga nad wykorzystaniem CPU w kontekście projektu}




        \chapter{Otrzymanie danych z kamery zdarzeniowej}
\label{cha:kamera}

\section{Specyfikacja techniczna wybranej kamery}
\label{sec:tech}

\section{Konfiguracja sprzętu}
\label{sec:konfiguracja}

\section{Uzyskanie obrazu z danych}
\label{sec:obraz}

	\chapter{Algorytm detekcji przeszkód}
\label{cha:algorytm}



        \chapter{Symulacja działania}
\label{cha:symulacja}

\section{Symulacja w Isaac SIM}
\label{sec:sym_isaac}

\section{Testy}
\label{sec:testy}
        \chapter{Implementacja na FPGA}
\label{cha:fpga}

\section{Wykorzystywany układ FPGA}
\label{sec:plytka}

\section{Implementacja}
\label{sec:implementacja}



	\chapter{Podsumowanie}
\label{cha:podsumowanie}

% \section{Efekty pracy i uzyskane wyniki}
% \label{sec:wyniki}

%Podsumowanie - krótko co zostało zrobione
W ramach pracy nad projektem wykonane zostały:
\begin{itemize}
    \item Przegląd literatury -- zapoznano się z metodami detekcji i~unikania przeszkód przy wykorzystaniu kamer zdarzeniowych. Przeanalizowano algorytmy wykrywania obiektów w~wybranych artykułach i~na tej podstawie stworzono własne podejście do problemu. Przeanalizowano publikacje pomocne w~jego realizacji.
    \item Środowisko symulacyjne do testów SiL -- stworzono różne scenariusze testowe (sceny), umożliwiające sprawdzenie działania algorytmu w~dziennych lub nocnych warunkach na zróżnicowanych torach przeszkód. W środowisku umieszczono model drona, wyposażonego w~kamerę zdarzeniową i~rozwiązano problem sterowania nim.
    \item Algorytm detekcji przeszkód -- zaprojektowany dzięki analizie literatury sposób działania algorytmu zaimplementowano w~języku programowania Python. Przeprowadzono wstępne próby jego działania na zbiorach danych pochodzących z~kamery zdarzeniowej DAVIS240. Używając \textit{frameworka} ROS2, przetestowano algorytm w~symulacji.
\end{itemize}

%Czego nie udało się zrobić

\vspace{11px}
Niestety, nie wszystkie cele wymienione w~rozdziale \ref{sec:celePracy} zostały osiągnięte. Z~powodu ograniczonego czasu oraz rozbudowanej formy projektu, nie udało się podjąć próby implementacji i~przetestowania rozwiązania w~formule \textit{Hardware in the Loop} na platformie wbudowanej.

% \section{Wnioski}
% \label{sec:wnioski}

%Opis zalet i wad uzyskanego algorytmu
Na podstawie testów systemu, przedstawionych i~opisanych w~rozdziale \ref{sec:algorytm_wyniki}, można zdefiniować jego cechy:
\begin{itemize}
    \item Algorytm dobrze radzi sobie w~ograniczonych warunkach oświetleniowych,
    \item Błędy w działaniu algorytmu występują częściej w ograniczonych warunkach oświetleniowych. Zazwyczaj są to:
    \begin{itemize}
        \item Niewykrywanie obiektu na pojedynczych klatkach obrazu,
        \item Wykrywanie tylko części przeszkody,
        \item Wykrywanie jednego obiektu jako kilka mniejszych przeszkód.
    \end{itemize}
\end{itemize}

Dzięki testom \textit{Software in the Loop}, możliwe było wykrycie wielu błędów i~ich naprawa na wczesnym etapie pracy nad systemem detekcji obiektów -- podczas implementacji modelu programowego.

% Czemu to co nie działa nie działa i jak by to można zrobić lepiej
W czasie testów SiL stwierdzony został poważny problem z~wartościami odległości do wykrytych obiektów zwracanymi przez algorytm. Niestety okazały się one na tyle niedokładne i~zawierały tyle błędów, że ich użycie do wyznaczania pozycji obiektów w~przestrzeni 3D oraz ich prędkości okazało się niemożliwe. Mimo podjętych prób naprawy, zachowując przyjęte podejście (triangulacja z~użyciem jednej kamery zdarzeniowej), nie udało się tego problemu rozwiązać.

Możliwe przyczyny błędnego działania tej fazy algorytmu:
\begin{itemize}
    \item Niedokładne dane o~pozycji i~rotacji kamery,
    \item Błędy w~dopasowywaniu punktów charakterystycznych,
    \item Błędy w~procesie śledzenia obiektów,
    \item Zbyt małe przemieszczenie drona między dwiema kolejnymi klatkami.
\end{itemize}

Ponieważ zastosowanie triangulacji w przypadku pojedynczej kamery okazało się błędnym podejściem, w~celu realizacji zadania rozpoznawania głębi, konieczna może okazać się zmiana metody. Proponowane rozwiązania:
\begin{itemize}
    \item Ograniczenie wykrywania do obiektów o~znanym rozmiarze -- znacznie zmniejszy to potencjalne zastosowania systemu, ale pozwoli na otrzymywanie danych o~głębi bez wyposażania drona w~dodatkowe czujniki.
    \item Umieszczenie na dronie LiDAR-u, jako dodatkowego czujnika. W tym rozwiązaniu kamera zdarzeniowa pozwoli na skuteczniejsze wykrywanie obiektów, które szybko się poruszają oraz poprawi działanie w ciemności, a~LiDAR umożliwi zwiększenie dokładności oraz pozwoli na precyzyjne odczytywanie głębi. Minusem takiego rozwiązania jest wzrost ceny oraz stopnia skomplikowania systemu. Znacznie zwiększyłaby się złożoność obliczeniowa z~powodu dodatkowych danych do przetworzenia.
    \item Zastosowanie dwóch kamer zdarzeniowych w~układzie stereo, dzięki czemu możliwe będzie przeprowadzenie triangulacji w poprawny i dokładny sposób.  % Dodatkowa masa drugiej kamery jako wada?
    \item Wybór i zastosowanie w projekcie algorytmu estymacji głębi dedykowanego dla pojedynczych kamer zdarzeniowych. Takie rozwiązania można znaleźć w literaturze na przykład w artykułach \cite{EMVS} lub \cite{single_depth}. W porównaniu do prostej koncepcyjnie triangulacji, są to rozbudowane i bardziej złożone obliczeniowo systemy.
\end{itemize}

% \section{Plany rozwoju}
% \label{sec:plany}

% plany rozwoju na przyszłość, potencjalne aplikacje, w których można zastosować algorytm itp.

\noindent Projekt ma szerokie możliwości dalszego rozwoju i~rozbudowy.

W ramach dalszej pracy nad projektem planowane jest:
\begin{itemize}
    \item Rozwiązanie problemu otrzymywania poprawnych danych o~głębi,
    \item Optymalizacja akumlacji zdarzeń do dalszego przetwarzania. Dwa podejścia do tego zagadnienia - zbieranie danych w pewnym przedziale czasowym oraz akumulacja pewnej ich liczby, można połączyć w jedno. Ten sposób pozwala na połączenie ich zalet i minimalizacje ich wad (odpowiednio możliwości wystąpienia rozmycia ruchu oraz za dużej do realizacji w czasie rzeczywistym częstotliwości ramek zdarzeniowych).
    \item Zaimplementowanie algorytmu na wbudowanej platformie obliczeniowej -- eGPU Jetson i~przetestowanie jego działania metodą HiL,
    \item Zaimplementowanie algorytmu na platformie z układem FPGA (ang. \textit{Field Programmable Gate Array}). Przeprowadzenie testów i~porównanie działania systemu uruchamianego na FPGA i~eGPU Jetson. FPGA jest platformą, która może okazać się najskuteczniejsza dla stworzonego algorytmu ze względu na szerokie możliwości zrównoleglania wykonywanych obliczeń, co w~przypadku systemów wizyjnych jest wyjątkowo efektywne.
    \item Zaprojektowanie i~implementacja systemu sterowania dronem, tak by na podstawie danych otrzymywanych z~procesu detekcji, umożliwić unikanie przeszkód,
    \item Zamontowanie systemu na rzeczywistym dronie i~przetestowanie go na przygotowanym torze przeszkód.
\end{itemize}

Mimo że algorytm detekcji testowany był z~wykorzystaniem czterowirnikowego drona, to może być zastosowany do wykrywania poruszających się względem kamery obiektów na innych rodzajach pojazdów i~wszelkich robotach mobilnych.

	
	% itd.
	% \appendix
	% \include{dodatekA}
	% \include{dodatekB}
	% itd.
	
	\printbibliography

\end{document}
